\documentclass[11pt,parskip]{scrartcl}
\usepackage[ngerman]{babel}

% For XeTeX %%%%%%%%%%%%%%%%%%%%%%%%%%%%%%%%%%%%%%%%%%%%%%%%%%%%%%%%%%%%%%%%%%%%
\usepackage{fontspec}
\setromanfont{Linux Libertine O}
\setsansfont{Linux Biolinum O}

% For pdfTeX %%%%%%%%%%%%%%%%%%%%%%%%%%%%%%%%%%%%%%%%%%%%%%%%%%%%%%%%%%%%%%%%%%%
%\usepackage[T1]{fontenc}
%\usepackage{inputenc}
%\usepackage{mathpazo}

% Packages %%%%%%%%%%%%%%%%%%%%%%%%%%%%%%%%%%%%%%%%%%%%%%%%%%%%%%%%%%%%%%%%%%%%%
%\usepackage{biblatex}
\usepackage{amsmath}
\usepackage[a4paper]{geometry}
\usepackage{graphicx}
\usepackage{xcolor}
\usepackage{microtype}
\usepackage{booktabs}
\usepackage[colorlinks=false, pdfborder={0 0 0 }]{hyperref}
\usepackage{cleveref}
\usepackage[autostyle=true,german=quotes]{csquotes}
\usepackage{blindtext}
%\usepackage{listings}
\usepackage{tcolorbox}
\tcbuselibrary{listings}
\usepackage{siunitx}

% Options %%%%%%%%%%%%%%%%%%%%%%%%%%%%%%%%%%%%%%%%%%%%%%%%%%%%%%%%%%%%%%%%%%%%%%
\widowpenalty=10000
\clubpenalty=10000

\definecolor{light-gray}{gray}{0.95}
\tcbset{
    listing only,
    colback=light-gray,
    coltext=black,
    boxrule=0pt,
    left=-2.5pt
  }


% Begin of document %%%%%%%%%%%%%%%%%%%%%%%%%%%%%%%%%%%%%%%%%%%%%%%%%%%%%%%%%%%%
\begin{document}


% Titlepage %%%%%%%%%%%%%%%%%%%%%%%%%%%%%%%%%%%%%%%%%%%%%%%%%%%%%%%%%%%%%%%%%%%%
%
\begin{titlepage}
  \begin{sffamily}
    {\scshape\LARGE \textcolor{gray}{Universität Hamburg}\par}
    {\scshape\Large \textcolor{gray}{Projektarbeit Interactive Visual
        Computing}\par}
    \vspace{2.5cm}
    \centering
    {\huge\bfseries Das wunderbare Leben des kleinen Tux\par}
    {\large\bfseries Ein POV-Ray-Animationsfilm\par}
    \vspace{1.5cm}
    {\Large\itshape \textcolor{darkgray}{
        Inga Lemme \quad{}
        Thomas Ort \quad{}
        Melanie Remmels
      }\par
    }
    \vfill
    \includegraphics[width=0.25\textwidth]{./fig/tuxegg}\par\vspace{1cm}
    \vfill
    % Bottom of the page
    {\large \today\par}
  \end{sffamily}
\end{titlepage}
%
% End of Titlepage %%%%%%%%%%%%%%%%%%%%%%%%%%%%%%%%%%%%%%%%%%%%%%%%%%%%%%%%%%%%%


\newpage
\tableofcontents
\newpage


\section{Projektidee}
Vorlage für die Hauptfigur des hier beschriebenen Films ist das Maskottchen
des freien Kernels \emph{Linux}. Dieses stellt einen Pinguin dar und wird kurz
\emph{Tux} genannt. \emph{Tux} wurde im Jahr 1996 von \emph{Larry Ewing} mit
der Bildbearbeitungssoftware \emph{GIMP} entworfen und steht seitdem zur freien
Verfügung für die Gemeinde. Er darf nach Belieben verwendet und verändert
werden, solange auf Nachfrage sowohl Urheber\footnote{lewing@isc.tamu.edu} als
auch das verwendete Programm genannt wird. \cite{ewing}

Die Idee, dass das Logo ausgerechnet einem Pinguin nachempfunden wurde, stammt
von \emph{Linus Torvalds}, dem Gründer von Linux. Laut \emph{Jeff Ayers}, einem
Linuxentwickler, besitzt \emph{Torvalds} eine Affinität für
\enquote{\emph{flugunfähige, fette Wasservögel}}, sodass letztlich der Entwurf
von \emph{Ewing} übernommen wurde. Eine weitere Anekdote, die zur Auswahl des
Pinguins beigetragen hat stammt von einem Erlebnis \emph{Torvalds} in einem
Aquarium in Canberra, Australien. Dort wurde er von einem Pinguin gebissen und
soll seitdem mit der Krankheit \emph{Penguinitis} infiziert sein:
%
\begin{quote}
  \enquote{Penguinitis makes you stay awake at nights just thinking about
    penguins and feeling great love towards them.} \cite{tuxstory}
\end{quote}
%
Es gibt inzwischen unzählige Versionen des Maskottchens, in dieser
Arbeitsgruppe haben wir uns an einer modernen und jungen Version des Pinguins
orientiert wie in Abbildung \ref{fig:overlord59tux} zu sehen.
%
\begin{figure}[htbp]
  \centering
  \includegraphics[width=0.3\textwidth]{./fig/overlord59tux.pdf}
  \caption{
    Die Hauptfigur des Filmes wurde nach diesem Vorbild entwickelt. Das
    gezeigte Modell stammt vom Autor \emph{Overlord59} und wurde unter
    \emph{Creative Commons BY-NC-SA} veröffentlicht.
    \emph{
      (Quelle:
      \url{http://tux.crystalxp.net/de
        .id.1568-overlord59-overlord59-tux-g2.html)
      }
    }
  }
  \label{fig:overlord59tux}
\end{figure}
%
In diesem Projekt wurde das Programm \emph{POV-Ray} verwendet, um
dreidimensionale Figuren und Objekte, sowie kurze Animationssequenzen zu
erstellen. Die Animationssequenzen bestehen dabei aus mehreren Bildern, die
mittels \emph{FFmpeg} in ein Videoformat konvertiert und anschließend mit
\emph{iMovie} geschnitten und vertont wurden.


\subsection{Plot des Films}
Zu Beginn des Films ist zunächst ein Ei zu sehen, welches einen immer größeren
Riss bekommt. Der obere Teil des Eis bricht auf und letztlich schlüpft der Tux.
Der Babytux erscheint und springt aus der unteren Hälfte des Eis heraus.
Anschließend entdeckt der Pinguin, dass er Füße, Flügel und seinen Schwanz
bewegen kann.

Ein durch das Bild fliegendes Flugzeug soll andeuten, wie Tux die
Welt und dabei allerlei Sehenswürdigkeiten entdeckt. Diese werden anschließend
in einer Bildergalerie gezeigt, durch die der inzwischen herangewachsene Tux
hindurch läuft.

In der Abschlusssequenz genießt Tux seinen Lebensabend auf einem Schaukelstuhl
vor seinem Haus.


\subsection{Liste der POV-Ray-Module}

\begin{description}
  \item [environment.pov] Aufbau der Umgebung; enthält Himmel und Boden.
  \item [egg.pov] Geschlossenes Ei-Objekt.
  \item [tux.pov] Körper des Tux ohne Accessoires.
  \item [bow.pov] Pinke Schleife.
  \item [soother.pov] Schnuller.
  \item [assempledTux.pov] Zusammengebauter Tux mit Accessoires.
  \item [crack.pov] Objekt zum Erstellen eines Risses im Ei-Objekt.
  \item [tuxIsBorn.pov] Animation des wackelnden Eis und Animation der
    Risse; Tux erscheint im Ei mit animiertem Nuckeln am Schnuller.
  \item [babytuxDiscovers.pov] Animation der beweglichen Gliedmaßen des Tux.
  \item [flying.pov] Flugzeug, welches horizontal durch das Bild fliegt.
  \item [gallery.pov] Die Bildergalerie mit dem hindurch laufenden Tux.
  \item [oldTux.pov] Tux in einem Schaukelstuhl vor seinem Haus.
\end{description}


\newpage


\section{Statischer Aufbau der Figuren}
Im Folgenden wird beschrieben mittels welcher POV-Ray-Funktionen und Objekte die
einzelnen Figuren, Requisiten und Szenenbilder des Films erstellt wurden.


\subsection{Konstruktion der Umgebung}
In der Datei \texttt{environment.pov} wurden alle relevanten Objekte der Umgebung
festgelegt. Der Himmel wurde mittels \texttt{sphere} der Boden mittels
\texttt{plane} realisiert. Dabei wurde für den Himmel eine \texttt{color\_map}
eingesetzt um einen Farbverlauf herzustellen. Für eine unebene Struktur des
Bodens wurde das Pattern \texttt{bumps} verwendet.


\subsection{Aufbau des Pinguineis}
Die Grundstruktur des Eis wurde von der Vorgängergruppe (Teil 1) übernommen
und an unseren Film angepasst. Dazu wurden jeweils für den oberen Teil und den
unteren Teil des Eis ein weiteres etwas kleineres Ei-Objekt erzeugt und mittels
\texttt{difference} vom größeren Objekt abgezogen. Dadurch wird das Ei von
innen hohl. Damit das jeweilige Objekt auch als hohl erkannt wird, wurde das
innere Objekt minimal nach oben bzw. nach unten verschoben.
%
\begin{tcblisting}{}
    difference{
    object{ Egg_lowerpart }
    object{
      Egg_lowerpart
      translate <0, 0.1, 0>
      scale <0.9, 0.9, 0.9>
    }
  }
\end{tcblisting}


\subsection{Aufbau der Hauptfigur}

\subsubsection{Körper}
Zu allererst wurden die Proportionen als \texttt{declare}-Anweisung festgelegt.
Um die Proportionen unabhängig von der Größe des Tux gleich zu halten, wird
nur die Höhe \texttt{tuxheight} variabel gehalten. Die anderen Größen wie
\texttt{tuxwidth} oder \texttt{radiustummy} wurden mit \texttt{tuxheight}
verrechnet.

Der Grundaufbau des Tux besteht aus zwei Kugeln, in POV-Ray \texttt{sphere}
genannt. Einer unteren großen Kugel für den Unterleib und einer etwas Kleineren
für den Kopf oberhalb. Der Unterleib besteht zunächst aus \emph{einer}
Kugel. Zur Realisierung des weißen Bauches wurden zwei weitere
\texttt{sphere}-Objekte erstellt, deren Schnittmenge (\emph{intersection})
anschließend mit der großen Kugel vereinigt wurden (\emph{union}).
%
\begin{tcblisting}{}
  union{
    intersection{
      sphere{ 0, radiustummy }
      sphere{ 0, radiustummy }
      scale <0.6, 1.5, 0.25>
      translate <0, 0, -radiustummy + 0.1>
    }
    pigment{ White }
    sphere{
      0, radiustummy
      pigment{ Gray10 }
    }
  }
\end{tcblisting}
%
Die weiße Schnittmenge wurde mit den Funktionen \texttt{scale} und
\texttt{translate} so verschoben und skaliert, dass der vordere Teil der
Hauptkugel in den richtigen Proportionen weiß erscheint. Sowohl Kopf, als auch
Unterleib wurden in einer \texttt{declare}-Anweisung als \texttt{head} und
\texttt{tummy} global geltend gemacht. So können diese direkt angesprochen
wurden, ohne den Code immer wieder neu reproduzieren zu müssen.

Der Kopf des Pinguins wurde mit einer \texttt{sphere} generiert, die zwei
Drittel der Größe des Unterleibes beträgt. In den Kopf wurden Augen mit
schwarzen Pupillen eingelassen. Zunächst wurde die Pupille in einer
\texttt{declare}-Anweisung festgelegt. Sie besteht wie der Bauch aus der
Schnittmenge zweier Kugeln. Die beiden Augen wurden in \texttt{LeftEye} und
\texttt{RightEye} deklariert. Hier wurden die Pupillen als Objekt mit einem
weiteren \texttt{sphere}-Objekt vereinigt (\texttt{union}).

Der Schnabel des Tux wurde mittels eines \texttt{cone}-Objektes realisiert.
Hierbei wurden Zentrum und Radius der beiden Enden, sowie die Skalierung des
gesamten Objektes so gewählt, dass ein flach gedrückter Kegel entsteht.

\subsubsection{Gliedmaßen}
Der Tux besteht weiterhin aus zwei Füßen und zwei Flügeln. Die Flügel bestehen
aus jeweils einem Objekt \texttt{Wing}, welches aus einer Differenz aus
\texttt{cone} und \texttt{sphere} gebildet wurde. Die Füße bestehen aus dem
Objekt \texttt{Foot}, der Schnittmenge aus \texttt{sphere} und \texttt{box}.
Dadurch wurde die Sphäre halbiert und es ist nur die obere Hälfte sichtbar.

In der Rückansicht ist ein Schwanz zu sehen. Dieser wurde aus einer einfachen
\texttt{cone} in passender Größe generiert und lässt sich als
\texttt{Tail}-Objekt ansprechen.

\subsubsection{Accessoires}
Der Tux ist in diesem Film ein weibliches Jungtier, daher wurde eine Schleife
(\texttt{Bow}) und ein Schnuller (\texttt{Soother}) konstruiert. Die Schleife
wurde zunächst als ein \texttt{PartBow}-Objekt deklariert, welches eine
\texttt{cone} generiert. In einer \texttt{union} wurden anschließend zwei dieser
Objekte zusammengefügt, wobei eines um \ang{180} gedreht wurde. Der Knoten der
Schleife wurde innerhalb der \texttt{union} mittels \texttt{sphere} umgesetzt.

Für den Schnuller wurde eine \texttt{sphere} und ein \texttt{torus} vereinigt.
Die beiden Accessoires wurden in jeweils eine Datei ausgelagert, dadurch sind
die Objekte vom eigentlichen Körper unabhängig und können bei Bedarf auch
weggelassen werden.


\newpage


\section{Aufbau der einzelnen Animationssequenzen}
Der überwiegende Teil der einzelnen Sequenzen wurde jeweils innerhalb einer
Datei mittels der Funktion \texttt{clock} und mehreren
\texttt{if/elsif/else}-Statements realisiert. Dazu wurde zu Beginn die Variable
\texttt{MyClock} deklariert mit den jeweiligen Zeiteinheiten verglichen, um
anschließend die gewünschte Bewegung festzulegen.
%
\begin{tcblisting}{}
  #declare My_Clock = Start + (End - Start) * clock;

  #if (My_Clock <= 1)
  /* tue etwas */

  #elseif(My_Clock <=2)
  /* tue etwas anderes */
  ...
\end{tcblisting}
%
Mittels dieser Deklarationen war es möglich die folgenden kurzen Sequenzen zu
erstellen.

\subsection{Anfangssequenz: Erstes Anzeichen von Leben}
Das Ei beginnt sich mehrere Male hin und her zu bewegen. Dazwischen gibt es
immer wieder Pausen. Für das Wackeln des Eis wurde die sinus-Funktion verwendet.
Diese wurde mit \texttt{MyClock} verknüpft und in eine \texttt{rotate}-Funktion
integriert. Dadurch bewegt sich das Ei in Abhängigkeit der Zeit hin und her.

\subsection{Sequenz 1: Schlüpfen des Tux}
Zunächst bekommt das Ei einen Riss, welcher immer größer wird und sich um das
ganze Ei ausdehnt. Sobald die obere Hälfte von der Unteren des Eis komplett
getrennt ist, hebt die obere Hälfte ab und fliegt nach hinten weg.

Zur Realisierung dieser Szene wurde ein Objekt \texttt{Crack} erstellt.
\texttt{Crack} besteht aus mehreren quadratischen \texttt{box}-Objekten die
ineinander verdreht wurden.
%
\begin{tcblisting}{}
  union{
    box{
      <-0.5, 0, -0.5>
      <0.5, 0.1, 0.5>
    }
    box{
      <-0.5, 0, -0.5>
      <0.5, 0.1, 0.5>
      rotate <5, 20, 10>
    }
    ...
  }
\end{tcblisting}
%
Dieses Objekt wurde so klein erstellt, dass es in das Ei-Objekt passt.
Anschließend wurde das \texttt{crack}-Objekt abhängig von der Zeit in der x- und
z-Richtung größer skaliert und eine Differenz zum Ei-Objekt gebildet. Die
Animation zeigt im Ergebnis einen Riss, der sich immer mehr vergrößert und am
Ende zwei Ei-Hälften erscheinen.

\subsection{Sequenz 2: Babytux entdeckt seine beweglichen Gliedmaßen}
Wie eingangs erwähnt wurden hier je nach Zeiteinheit die einzelnen Gliedmaßen
bewegt. Dazu wurden die einzelnen Gliedmaßen aus der Datei \texttt{tux.pov}
angesprochen. Diese wurden anschließend mit den Funktionen \texttt{rotate} oder
\texttt{translate} bewegt, wobei diese mit der Funktion \texttt{clock}
verknüpft wurden.

Dies ist auch die erste Sequenz, in welcher eine Kamerafahrt realisiert wurde.
Dazu wurde die Funktion \texttt{camera} wie ein Objekt behandelt und zu den
jeweiligen Zeiteinheiten entsprechend rotiert und translatiert.


\subsection{Sequenz 3: Flugzeug}
Das Flugzeug wurde aus den Beispielgalerien von Friedrich A. Lohmüller
\cite{f-lohmueller} verwendet.  Dazu wurde die Datei \texttt{Plane\_00.inc}
inkludiert und mit den Funktionen \texttt{rotate}, \texttt{scale} und
\texttt{translate} an die Bedürfnisse angepasst. Zur Animation wurde das Objekt
mittels \texttt{clock} translatiert.


\subsection{Sequenz 4: Bildergalerie}
Zur Realisierung der Bildergalerie wurden verschiedene Bilddateien an das Makro
\texttt{PictureFrame} übergeben, welches in der Datei \texttt{pictureFrame.pov}
implementiert wurde. Dieses Makro wurde anschließend für jede Bilddatei im
Modul \texttt{gallery.pov} aufgerufen und entsprechend platziert. Dieses Modul
wurde wiederum neben \texttt{walkingTux.pov} in die eigentliche
Animationssequenz \texttt{bigTux.pov} inkludiert.


\subsection{Abschlusssequenz: Lebensabend}
In der Abschlusssequenz wurde der blaue Himmel mittels dem Modul
\texttt{stars.inc} in einen Sternenhimmel geändert. Als Textur der
\texttt{sphere} wurde \texttt{Starfield1} gewählt und skaliert. Bis auf Tux
wurden alle Objekte von Friedrich A. Lohmüller \cite{f-lohmueller} verwendet
und angepasst. So wurden beispielsweise die Wände des Haus-Objektes weiß
gefärbt. Dazu sind die entsprechenden Objekte wieder als Makros implementiert.

Dem Objekt Schaukelstuhl musste ein Winkel übergeben werden. Um ein Schaukeln
zu simulieren, wurde dieser mit der Sinusfunktion und \texttt{clock} verknüpft.
Tux wurde mit dem Schaukelstuhl in einer \texttt{union} vereinigt.


\newpage


\section{Fertigstellung des gesamten Films}
Alle Module, die Animationssequenzen enthielten wurden mittels \emph{FFmpeg} in
Videoclips konvertiert. Das Kommandozeilenprogramm \emph{FFmpeg} besitzt
zahlreiche Parameter, welche Bildqualität und Format beeinflussen. In diesem
Projekt wurden aus den einzelnen Bildern mp4-Dateien mit folgenden Parametern
erstellt:
%
\begin{tcblisting}{}
  ffmpeg -framerate 24 -i DATEIEN.png -c:v libx264 -r 30 -pix_fmt yuv420p
  MOVIE.mp4
\end{tcblisting}
%

\subsection{Zusammenfügen der einzelnen Sequenzen}
Das Zusammenfügen der Animationssequenzen war mittels \emph{iMovie} intuitiv
möglich. Dazu wurden die Filme einfach per Drag and Drop in die Zeitleiste
gezogen. Da die Geschwindigkeiten alle von der Funktion \texttt{clock}
abhingen, mussten diese noch entsprechend beschleunigt bzw. verlangsamt werden.

Um den Übergang zwischen den einzelnen Clips zu optimieren, wurden die
\emph{iMovie-Übergangsstile} verwendet. Zum Anzeigen von kurzen Texten wurde
die Titel-Funktion verwendet.


\subsection{Vertonung}
Der komplette Film ist mit Musik unterlegt. Dabei wechselt die Musik zu den
jeweiligen Abschnitten. Für einen fließenden Übergang wurde die Fade-Funktion
verwendet. Zu bestimmten Zeitpunkten sind Soundeffekte eingefügt, die alle aus
der Bibliothek aus \emph{iMovie} stammen.


\subsection{Korrekturen}
Übergänge, die nicht korrekt geschnitten werden konnten wurden durch
Einzelbilder ersetzt, die in das Programm \emph{iMovie} zum jeweiligen
Zeitpunkt eingefügt wurden.


\newpage


\begin{figure}[htbp]
  \centering
  \includegraphics[width=0.95\textwidth]{./fig/poster.jpg}
  \caption{
    Poster zum POV-Ray-Animationsfilm \enquote{Das wunderbare Leben des kleinen
      Tux}.
  }
  \label{fig:poster}
\end{figure}


\newpage


\addcontentsline{toc}{section}{Literatur}
\bibliographystyle{unsrt}
\bibliography{bibtux}

\end{document}
