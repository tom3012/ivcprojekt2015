\documentclass[11pt,parskip]{scrartcl}
\usepackage[ngerman]{babel}

% For XeTeX %%%%%%%%%%%%%%%%%%%%%%%%%%%%%%%%%%%%%%%%%%%%%%%%%%%%%%%%%%%%%%%%%%%%
\usepackage{fontspec}
\setromanfont{Linux Libertine O}
\setsansfont{Linux Biolinum O}

% For pdfTeX %%%%%%%%%%%%%%%%%%%%%%%%%%%%%%%%%%%%%%%%%%%%%%%%%%%%%%%%%%%%%%%%%%%
%\usepackage[T1]{fontenc}
%\usepackage{inputenc}
%\usepackage{mathpazo}

% Packages %%%%%%%%%%%%%%%%%%%%%%%%%%%%%%%%%%%%%%%%%%%%%%%%%%%%%%%%%%%%%%%%%%%%%
%\usepackage{biblatex}
\usepackage{amsmath}
\usepackage[a4paper]{geometry}
\usepackage{graphicx}
\usepackage{xcolor}
\usepackage{microtype}
\usepackage{booktabs}
\usepackage[colorlinks=false, pdfborder={0 0 0 }]{hyperref}
\usepackage{cleveref}
\usepackage[autostyle=true,german=quotes]{csquotes}
\usepackage{blindtext}

% Options %%%%%%%%%%%%%%%%%%%%%%%%%%%%%%%%%%%%%%%%%%%%%%%%%%%%%%%%%%%%%%%%%%%%%%
\widowpenalty=10000
\clubpenalty=10000


% Begin of document %%%%%%%%%%%%%%%%%%%%%%%%%%%%%%%%%%%%%%%%%%%%%%%%%%%%%%%%%%%%
\begin{document}


% Titlepage %%%%%%%%%%%%%%%%%%%%%%%%%%%%%%%%%%%%%%%%%%%%%%%%%%%%%%%%%%%%%%%%%%%%
%
\begin{titlepage}
  \begin{sffamily}
    {\scshape\LARGE \textcolor{gray}{Universität Hamburg}\par}
    {\scshape\Large \textcolor{gray}{Projektarbeit Interactive Visual
        Computing}\par}
    \vspace{2.5cm}
    \centering
    {\huge\bfseries Babytux\par}
    {\large\bfseries Ein Povray-Film über das Erwachsenwerden eines Pinguins\par}
    \vspace{1.5cm}
    {\Large\itshape \textcolor{darkgray}{
        Lemme, Inga \quad{}
        Ort, Thomas \quad{}
        Remmels, Melanie
      }\par
    }
    \vfill
    \includegraphics[width=0.20\textwidth]{./fig/ourtux.pdf}\par\vspace{1cm}
    \vfill
    % Bottom of the page
    {\large \today\par}
  \end{sffamily}
\end{titlepage}
%
% End of Titlepage %%%%%%%%%%%%%%%%%%%%%%%%%%%%%%%%%%%%%%%%%%%%%%%%%%%%%%%%%%%%%


\newpage
\tableofcontents
\newpage


\section{Projektidee}
Vorlage für die Hauptfigur des hier beschriebenen Films ist das Maskottchen des
freien Kernels \emph{Linux}. Dieses stellt einen Pinguin dar und wird kurz
\emph{Tux} genannt. \emph{Tux} wurde im Jahr 1996 von \emph{Larry Ewing} mit
der Bildbearbeitungssoftware \emph{GIMP} entworfen und steht seitdem zur freien
Verfügung für die Gemeinde. Er darf nach Belieben verwendet und verändert
werden, solange auf Nachfrage sowohl Urheber\footnote{lewing@isc.tamu.edu} als
auch das verwendete Programm genannt werden. \cite{ewing}

Die Idee, dass das Logo ausgerechnet einem Pinguin nachempfunden ist, stammt
von \emph{Linus Torvalds}, dem Gründer von Linux. Laut \emph{Jeff Ayers}, einem
Linuxentwickler, besitzt \emph{Torvalds} eine Affinität für
\enquote{\emph{flugunfähige, fette Wasservögel}}, sodass letztlich der Entwurf
von \emph{Ewing} übernommen wurde. Eine weitere Anekdote, die zur Auswahl des
Pinguins beigetragen hat stammt von einem Erlebnis \emph{Torvalds} in einem
Aquarium in Canberra, Australien. Dort wurde er von einem Pinguin gebissen und
sei seitdem mit der Krankheit \emph{Penguinitis} infiziert:

\begin{quote}
  \enquote{Penguinitis makes you stay awake at nights just thinking about
    penguins and feeling great love towards them.} \cite{tuxstory}
\end{quote}

Es gibt inzwischen unzählige Versionen des Maskottchens, in dieser
Arbeitsgruppe haben wir uns an einer modernen und jungen Version des Pinguins
orientiert wie in Abbildung \ref{fig:taticetux} zu sehen.

\begin{figure}[htbp]
  \centering
  \includegraphics[width=0.3\textwidth]{./fig/tatice-g2-tux.pdf}
  \caption{
    Die Hauptfigur des Filmes wurde nach diesem Vorbild entwickelt. Das
    gezeigte Modell stammt vom Autor \emph{Tatice} und wurde unter
    \emph{Creative Commons BY-NC-SA} veröffentlicht.
    \emph{
      (Quelle: \url{http://tux.crystalxp.net/de.id.21103-tatice-g2-tux.html)}
    }
  }
  \label{fig:taticetux}
\end{figure}


\subsection{Plot des Films}
Zu Beginn des Films ist zunächst ein Ei zu sehen, welches immer größere Risse
bekommt und aus diesem letztlich der Tux schlüpft. Anschließend entdeckt der
Pinguin, dass er Füße, Flügel und seinen Schwanz bewegen kann. Nachdem er sich
etwas umsieht beginnt er seine Gegend zu erkunden und läuft los. Im weiteren
Verlauf wird ein Zeitraffer-Effekt eingesetzt. Der Tux wächst langsam und
entdeckt dabei die schönen und schwierigen Dinge des Lebens.


\newpage

\section{Statischer Aufbau der Figuren}
Im Folgenden wird beschrieben mittels welcher povray-Funktionen die einzelnen
Figuren und Objekte des Films erstellt wurden.


\subsection{Konstruktion der Umgebung}


\subsection{Aufbau eines Pinguineis}


\subsection{Aufbau der Hauptfigur}
Der Grundaufbau des Tux besteht aus zwei Kugeln in povray \texttt{sphere}
genannt.


\subsubsection{Bewegliche Gliedmaßen}


\newpage

\section{Aufbau der einzelnen Animationssequenzen}


\subsection{Titelsequenz}


\subsection{Sequenz 1: Schlüpfen des Tux}


\subsection{Sequenz 2: Babytux entdeckt seine beweglichen Gliedmaßen}


\subsection{Sequenz 3: Babytux macht erste Gehversuche}


\subsection{Abschlusssequenz}


\newpage

\section{Fertigstellung des gesamten Films}


\subsection{Zusammenfügen der einzelnen Sequenzen}


\subsection{Vertonung}


\subsection{Korrekturen}


\newpage
\addcontentsline{toc}{section}{Literatur}
\bibliographystyle{unsrt}
\bibliography{bibtux}

\end{document}
